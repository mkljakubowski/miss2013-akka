\documentclass{beamer}
\usepackage{graphicx}
\usepackage{listings}
\usepackage{color}
\usetheme{Warsaw}
\setbeamercovered{invisible}
%\usepackage{bm}         % For typesetting bold math (not \mathbold)
%\logo{\includegraphics[height=0.6cm]{yourlogo.eps}}
\title{Browser-based MMO Game with Akka and Play!}
%\subtitle{In one hour}
\author{Piotr Kukielka}
\date{\today}

% "define" Scala
\lstdefinelanguage{scala}{
  morekeywords={abstract,case,catch,class,def,%
    do,else,extends,false,final,finally,%
    for,if,implicit,import,match,mixin,%
    new,null,object,override,package,%
    private,protected,requires,return,sealed,%
    super,this,throw,trait,true,try,%
    Async, Action,%
    type,val,var,while,with,yield},
  otherkeywords={=>,<-,<\%,<:,>:,\#,@},
  sensitive=true,
  morecomment=[l]{//},
  morecomment=[n]{/*}{*/},
  morestring=[b]",
  morestring=[b]',
  morestring=[b]"""
}

\definecolor{dkgreen}{rgb}{0,0.6,0}
\definecolor{gray}{rgb}{0.5,0.5,0.5}
\definecolor{mauve}{rgb}{0.58,0,0.82}

% Default settings for code listings
\lstset{frame=tb,
  language=scala,
  aboveskip=3mm,
  belowskip=3mm,
  showstringspaces=false,
  columns=flexible,
  basicstyle={\small\ttfamily},
  numbers=none,
  numberstyle=\tiny\color{gray},
  keywordstyle=\color{blue},
  commentstyle=\color{dkgreen},
  stringstyle=\color{mauve},
  frame=single,
  breaklines=true,
  breakatwhitespace=true
  tabsize=3
}

\begin{document}

\begin{frame}
	\titlepage
\end{frame}

\begin{center}
	\tableofcontents
\end{center}

\section{Introduction}
	\subsection{What are we building?}
		\begin{frame}
			\begin{center}
					\includegraphics[scale=0.19]{gameplay}

					Source code: https://github.com/pkukielka/QuickPath
			\end{center}
		\end{frame}

	\subsection{About Play}
		\begin{frame}
			\begin{block}{Why Play?}
				\begin{itemize}
					\item Nice integration with Akka
					\pause \item Out of the box support for WebSockets and streams
					\pause \item Embedded support for various assets (i.e. CoffeeScript, LESS)
					\pause \item Simplified change-compile-run cycle
					\pause \item Easy to work with Json
				\end{itemize}
			\end{block}
		\end{frame}
		\begin{frame}
			\begin{block}{Build system}
				\begin{itemize}
					\item Play console, sbt
				\end{itemize}
			\end{block}
			\begin{block}{Project structure - standard application layout}
				\begin{itemize}
					\pause \item MVC architecture (app/ directory)
					\pause \item Resources (public/ directory)
					\pause \item Configuration (conf/ directory)
					\pause \item Unmanaged library dependencies (lib/ directory)
					\pause \item Build definitions (project/ directory)
					\pause \item Output files (target/ directory)
					\pause \item Tests (test/ directory)
					\pause \item Logs (log/ directory)
				\end{itemize}
			\end{block}
		\end{frame}

\section{Async in Play}
	\subsection{AsyncResult}
		\begin{frame}[fragile]
			Expensive tasks should be performed in separate threads using Future[Result].
			Async \{ \} is an helper method that builds an AsyncResult from a Future[Result].


			\begin{lstlisting}
				def index = Action {
				  val futureInt = scala.concurrent.Future { intensiveComputation() }
				  Async {
				    futureInt.map(i => Ok("Got result: " + i))
				  }
				}
			\end{lstlisting}
		\end{frame}

	\subsection{WebSockets and Comet}
		\begin{frame}[fragile]
			Use WebSockets instead of Comet when possible!

			\begin{lstlisting}
				def index = WebSocket.using[String] { request =>
				  // Log events to the console
				  val in = Iteratee.foreach[String](println).mapDone {
				    _ => println("Disconnected")
				  }

				  // Send a single 'Hello!' message
				  val out = Enumerator("Hello!")

				  (in, out)
				}
			\end{lstlisting}

			\pause Let's see it in practice!
		\end{frame}

\section{Iteratees, Enumerators and Enumeratees}
	\subsection{Iteratees}
		\begin{frame}[fragile]
			What iteratee do?
			\begin{itemize}
				\pause \item iterate over an Enumerator (iteratee is consumer)
				\pause \item accept static typed chunks and compute a static typed result (chunk by chunk)
				\pause \item can propagate the immutable context and state over iterations
			\end{itemize}

			\pause \begin{lstlisting}
				val sum: Iteratee[Int,Int] =
			    Iteratee.fold[Int,Int](0) { (s, e) => s + e }
			\end{lstlisting}

			\pause \begin{block}{Different look..}
				Iteratee is just a state machine in charge of looping over state Cont until it detects conditions to switch to terminal states Done or Error.
			\end{block}
		\end{frame}

	\subsection{Enumerators}
		\begin{frame}[fragile]
			What iteratee do?
			\begin{itemize}
				\pause \item produces statically typed chunks of data
				\pause \item requires a consumer to produce
			\end{itemize}

			\pause \begin{lstlisting}
				val integerEnumerator: Enumerator[Int] =
				  Enumerate(45, 33, 25, 45)

				val fileEnumerator: Enumerator[Array[Byte]] =
				  Enumerator.fromFile("some_file.txt")

				val dateGenerator: Enumerator[String] =
					Enumerator.generateM(
					  play.api.libs.concurrent.Promise.timeout(
					    Some(12345),
					    500
					  )
					)
			\end{lstlisting}
		\end{frame}

	\subsection{Enumeratees}
		\begin{frame}[fragile]
			Enumeratee is kind of pipe between between Enumerator and Iteratee.
			\begin{itemize}
				\pause \item it can be applied to an Enumerator without Iteratee
				\pause \item it can transform an Iteratee
				\pause \item it can be composed with other Enumeratee
			\end{itemize}

			\pause \begin{lstlisting}
				val enumerator = Enumerator(123, 345, 456)
				val iteratee: Iteratee[String, List[String]] = ???

				val list: List[String] =
				  enumerator through Enumeratee.map(_.toString) run iteratee
			\end{lstlisting}

		\end{frame}

\section{Coding!}
	\subsection{Coding!}
		\begin{frame}
			\begin{block}{What now?}
				\begin{itemize}
					\pause \item Iteratees with WebSockets in practice
					\pause \item Communication between multiple actors with Akka
					\pause \item Json serialization/deserialization
				\end{itemize}
			\end{block}
		\end{frame}

\end{document}